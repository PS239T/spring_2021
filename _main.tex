% Options for packages loaded elsewhere
\PassOptionsToPackage{unicode}{hyperref}
\PassOptionsToPackage{hyphens}{url}
%
\documentclass[
]{book}
\usepackage{lmodern}
\usepackage{amssymb,amsmath}
\usepackage{ifxetex,ifluatex}
\ifnum 0\ifxetex 1\fi\ifluatex 1\fi=0 % if pdftex
  \usepackage[T1]{fontenc}
  \usepackage[utf8]{inputenc}
  \usepackage{textcomp} % provide euro and other symbols
\else % if luatex or xetex
  \usepackage{unicode-math}
  \defaultfontfeatures{Scale=MatchLowercase}
  \defaultfontfeatures[\rmfamily]{Ligatures=TeX,Scale=1}
\fi
% Use upquote if available, for straight quotes in verbatim environments
\IfFileExists{upquote.sty}{\usepackage{upquote}}{}
\IfFileExists{microtype.sty}{% use microtype if available
  \usepackage[]{microtype}
  \UseMicrotypeSet[protrusion]{basicmath} % disable protrusion for tt fonts
}{}
\makeatletter
\@ifundefined{KOMAClassName}{% if non-KOMA class
  \IfFileExists{parskip.sty}{%
    \usepackage{parskip}
  }{% else
    \setlength{\parindent}{0pt}
    \setlength{\parskip}{6pt plus 2pt minus 1pt}}
}{% if KOMA class
  \KOMAoptions{parskip=half}}
\makeatother
\usepackage{xcolor}
\IfFileExists{xurl.sty}{\usepackage{xurl}}{} % add URL line breaks if available
\IfFileExists{bookmark.sty}{\usepackage{bookmark}}{\usepackage{hyperref}}
\hypersetup{
  pdftitle={Computational Thinking for Social Scientists},
  pdfauthor={Jae Yeon Kim},
  hidelinks,
  pdfcreator={LaTeX via pandoc}}
\urlstyle{same} % disable monospaced font for URLs
\usepackage{color}
\usepackage{fancyvrb}
\newcommand{\VerbBar}{|}
\newcommand{\VERB}{\Verb[commandchars=\\\{\}]}
\DefineVerbatimEnvironment{Highlighting}{Verbatim}{commandchars=\\\{\}}
% Add ',fontsize=\small' for more characters per line
\usepackage{framed}
\definecolor{shadecolor}{RGB}{248,248,248}
\newenvironment{Shaded}{\begin{snugshade}}{\end{snugshade}}
\newcommand{\AlertTok}[1]{\textcolor[rgb]{0.94,0.16,0.16}{#1}}
\newcommand{\AnnotationTok}[1]{\textcolor[rgb]{0.56,0.35,0.01}{\textbf{\textit{#1}}}}
\newcommand{\AttributeTok}[1]{\textcolor[rgb]{0.77,0.63,0.00}{#1}}
\newcommand{\BaseNTok}[1]{\textcolor[rgb]{0.00,0.00,0.81}{#1}}
\newcommand{\BuiltInTok}[1]{#1}
\newcommand{\CharTok}[1]{\textcolor[rgb]{0.31,0.60,0.02}{#1}}
\newcommand{\CommentTok}[1]{\textcolor[rgb]{0.56,0.35,0.01}{\textit{#1}}}
\newcommand{\CommentVarTok}[1]{\textcolor[rgb]{0.56,0.35,0.01}{\textbf{\textit{#1}}}}
\newcommand{\ConstantTok}[1]{\textcolor[rgb]{0.00,0.00,0.00}{#1}}
\newcommand{\ControlFlowTok}[1]{\textcolor[rgb]{0.13,0.29,0.53}{\textbf{#1}}}
\newcommand{\DataTypeTok}[1]{\textcolor[rgb]{0.13,0.29,0.53}{#1}}
\newcommand{\DecValTok}[1]{\textcolor[rgb]{0.00,0.00,0.81}{#1}}
\newcommand{\DocumentationTok}[1]{\textcolor[rgb]{0.56,0.35,0.01}{\textbf{\textit{#1}}}}
\newcommand{\ErrorTok}[1]{\textcolor[rgb]{0.64,0.00,0.00}{\textbf{#1}}}
\newcommand{\ExtensionTok}[1]{#1}
\newcommand{\FloatTok}[1]{\textcolor[rgb]{0.00,0.00,0.81}{#1}}
\newcommand{\FunctionTok}[1]{\textcolor[rgb]{0.00,0.00,0.00}{#1}}
\newcommand{\ImportTok}[1]{#1}
\newcommand{\InformationTok}[1]{\textcolor[rgb]{0.56,0.35,0.01}{\textbf{\textit{#1}}}}
\newcommand{\KeywordTok}[1]{\textcolor[rgb]{0.13,0.29,0.53}{\textbf{#1}}}
\newcommand{\NormalTok}[1]{#1}
\newcommand{\OperatorTok}[1]{\textcolor[rgb]{0.81,0.36,0.00}{\textbf{#1}}}
\newcommand{\OtherTok}[1]{\textcolor[rgb]{0.56,0.35,0.01}{#1}}
\newcommand{\PreprocessorTok}[1]{\textcolor[rgb]{0.56,0.35,0.01}{\textit{#1}}}
\newcommand{\RegionMarkerTok}[1]{#1}
\newcommand{\SpecialCharTok}[1]{\textcolor[rgb]{0.00,0.00,0.00}{#1}}
\newcommand{\SpecialStringTok}[1]{\textcolor[rgb]{0.31,0.60,0.02}{#1}}
\newcommand{\StringTok}[1]{\textcolor[rgb]{0.31,0.60,0.02}{#1}}
\newcommand{\VariableTok}[1]{\textcolor[rgb]{0.00,0.00,0.00}{#1}}
\newcommand{\VerbatimStringTok}[1]{\textcolor[rgb]{0.31,0.60,0.02}{#1}}
\newcommand{\WarningTok}[1]{\textcolor[rgb]{0.56,0.35,0.01}{\textbf{\textit{#1}}}}
\usepackage{longtable,booktabs}
% Correct order of tables after \paragraph or \subparagraph
\usepackage{etoolbox}
\makeatletter
\patchcmd\longtable{\par}{\if@noskipsec\mbox{}\fi\par}{}{}
\makeatother
% Allow footnotes in longtable head/foot
\IfFileExists{footnotehyper.sty}{\usepackage{footnotehyper}}{\usepackage{footnote}}
\makesavenoteenv{longtable}
\usepackage{graphicx}
\makeatletter
\def\maxwidth{\ifdim\Gin@nat@width>\linewidth\linewidth\else\Gin@nat@width\fi}
\def\maxheight{\ifdim\Gin@nat@height>\textheight\textheight\else\Gin@nat@height\fi}
\makeatother
% Scale images if necessary, so that they will not overflow the page
% margins by default, and it is still possible to overwrite the defaults
% using explicit options in \includegraphics[width, height, ...]{}
\setkeys{Gin}{width=\maxwidth,height=\maxheight,keepaspectratio}
% Set default figure placement to htbp
\makeatletter
\def\fps@figure{htbp}
\makeatother
\setlength{\emergencystretch}{3em} % prevent overfull lines
\providecommand{\tightlist}{%
  \setlength{\itemsep}{0pt}\setlength{\parskip}{0pt}}
\setcounter{secnumdepth}{5}
\usepackage{booktabs}
\usepackage{amsthm}
\makeatletter
\def\thm@space@setup{%
  \thm@preskip=8pt plus 2pt minus 4pt
  \thm@postskip=\thm@preskip
}
\makeatother
\usepackage[]{natbib}
\bibliographystyle{apalike}

\title{Computational Thinking for Social Scientists}
\author{Jae Yeon Kim}
\date{2020-09-27}

\begin{document}
\maketitle

{
\setcounter{tocdepth}{1}
\tableofcontents
}
\hypertarget{ps239t}{%
\chapter{PS239T}\label{ps239t}}

Welcome to PS239T

\begin{center}\rule{0.5\linewidth}{0.5pt}\end{center}

This course will help social science graduate students to think computationally and develop proficiency with computational tools and techniques, necessary to conduct research in computational social science. Mastering these tools and techniques not only enables students to collect, wrangle, analyze, and interpret data with less pain and more fun, but it also let students to work on research projects that would previously seem impossible.

The course is currently divided into two main subjects (fundamentals and applications) and six main sessions.

\textbf{Part I Fundamentals}

\begin{itemize}
\item
  In the first section, students learn best practices in data and code management using Git and Bash.
\item
  In the second, students learn how to wrangle, model, and visualize data easier and faster.
\item
  In the third, students learn how to use functions to automate repeated things and develop their own data tools (e.g., packages).
\end{itemize}

\textbf{Part II Applications}

\begin{itemize}
\item
  In the fourth, students learn how to collect and parse semi-structured data at scale (e.g., using APIs and webscraping).
\item
  In the fifth, students learn how to analyze high-dimensional data (e.g., text) using machine learning.
\item
  In the final, students learn how to access, query, and manage big data using SQL.
\end{itemize}

\textbf{Instructor and course developer}

\href{https://jaeyk.github.io/}{Jae Yeon Kim}: \href{mailto:jaeyeonkim@berkeley.edu}{\nolinkurl{jaeyeonkim@berkeley.edu}}

\textbf{History}

This course is a remix version of \href{https://github.com/rochelleterman/PS239T}{the course} originally developed by \href{https://github.com/rochelleterman}{Rochelle Terman}.

\textbf{Questions, comments, or suggestions}

Please \href{https://github.com/jaeyk/PS239T/issues}{create issues}, if you have questions, comments, or suggestions.

\includegraphics{https://i.creativecommons.org/l/by/4.0/88x31.png} This work is licensed under a \href{https://creativecommons.org/licenses/by/4.0/}{Creative Commons Attribution 4.0 International License}.

\hypertarget{Intro}{%
\chapter{Managing data and code}\label{Intro}}

\hypertarget{project-oriented-research}{%
\section{Project-oriented research}\label{project-oriented-research}}

\hypertarget{computational-reproducibility}{%
\subsection{Computational reproducibility}\label{computational-reproducibility}}

\hypertarget{setup}{%
\subsubsection{Setup}\label{setup}}

\begin{Shaded}
\begin{Highlighting}[]
\NormalTok{pacman}\OperatorTok{::}\KeywordTok{p\_load}\NormalTok{(}
\NormalTok{  tidyverse, }\CommentTok{\# tidyverse}
\NormalTok{  here }\CommentTok{\# computational reproducibility}
\NormalTok{)}
\end{Highlighting}
\end{Shaded}

\hypertarget{motivation}{%
\subsubsection{Motivation}\label{motivation}}

Why do you need to make your research project computationally reproducible?

For your self-interest and public benefits.

\includegraphics{https://github.com/dlab-berkeley/efficient-reproducible-project-management-in-R/blob/master/misc/screenshot.png?raw=true}

\hypertarget{how-to-organize-files-in-a-project}{%
\subsubsection{How to organize files in a project}\label{how-to-organize-files-in-a-project}}

You won't be able to reproduce your project unless it is efficiently organized.

Step 1. \href{https://environments.rstudio.com/}{\textbf{Environment}} is part of your project. If someone can't reproduce your environment, they won't be able to run your code.

\begin{itemize}
\tightlist
\item
  Launch R Studio. Choose Tools \textgreater{} Global Options. You should not check \texttt{Restor\ .RData\ into\ workspace\ at\ startup} and set saving workspace option to \texttt{NEVER}.
\end{itemize}

Step 2. For each project, create a project directory named after the project.

\begin{Shaded}
\begin{Highlighting}[]
\CommentTok{\# Don\textquotesingle{}t name it a project. Use a name that\textquotesingle{}s more informative. For instance, us\_election not my\_project.}

\KeywordTok{dir.create}\NormalTok{(}\StringTok{"../us\_election"}\NormalTok{)}
\end{Highlighting}
\end{Shaded}

Step 3. Launch R Studio. Choose File \textgreater{} New project \textgreater{} Browse existing directories \textgreater{} Create project This allows each project has its own workspace.

Step 4. Organize files by putting them in separate subdirectories and naming them in a sensible way.

\begin{itemize}
\tightlist
\item
  Treat raw data as read only (raw data should be RAW!) and put in the \texttt{data} directory.
\end{itemize}

\begin{Shaded}
\begin{Highlighting}[]
\KeywordTok{dir.create}\NormalTok{(here}\OperatorTok{::}\KeywordTok{here}\NormalTok{(}\StringTok{"us\_election"}\NormalTok{, }\StringTok{"data"}\NormalTok{))}
\end{Highlighting}
\end{Shaded}

\begin{itemize}
\tightlist
\item
  Separate read-only data from processed data and put in the \texttt{processed\_data} subdirectory.
\end{itemize}

\begin{Shaded}
\begin{Highlighting}[]
\KeywordTok{dir.create}\NormalTok{(here}\OperatorTok{::}\KeywordTok{here}\NormalTok{(}\StringTok{"us\_election"}\NormalTok{, }\StringTok{"processed\_data"}\NormalTok{))}
\end{Highlighting}
\end{Shaded}

\begin{itemize}
\tightlist
\item
  Put your code in the \texttt{src} directory.
\end{itemize}

\begin{Shaded}
\begin{Highlighting}[]
\KeywordTok{dir.create}\NormalTok{(here}\OperatorTok{::}\KeywordTok{here}\NormalTok{(}\StringTok{"us\_election"}\NormalTok{, }\StringTok{"src"}\NormalTok{))}
\end{Highlighting}
\end{Shaded}

\begin{itemize}
\tightlist
\item
  Put generated outputs (e.g., tables, figures) in the \texttt{outputs} subdirectory and treat them as disposable.
\end{itemize}

\begin{Shaded}
\begin{Highlighting}[]
\KeywordTok{dir.create}\NormalTok{(here}\OperatorTok{::}\KeywordTok{here}\NormalTok{(}\StringTok{"us\_election"}\NormalTok{, }\StringTok{"outputs"}\NormalTok{))}
\end{Highlighting}
\end{Shaded}

\begin{itemize}
\tightlist
\item
  Put your custom functions in the \texttt{functions} subdirectory. You can put some of them together later as a package.
\end{itemize}

\begin{Shaded}
\begin{Highlighting}[]
\KeywordTok{dir.create}\NormalTok{(here}\OperatorTok{::}\KeywordTok{here}\NormalTok{(}\StringTok{"us\_election"}\NormalTok{, }\StringTok{"functions"}\NormalTok{))}
\end{Highlighting}
\end{Shaded}

\textbf{Challenge 2}

Set a project structure for a project named ``starwars''.

\hypertarget{how-to-organize-code-in-a-r-markdown-file}{%
\subsubsection{How to organize code in a R markdown file}\label{how-to-organize-code-in-a-r-markdown-file}}

\begin{itemize}
\item
  In addition to environment, \textbf{workflow} is an important component of project efficiency and reproducibility.
\item
  R Markdown basic syntax
\end{itemize}

\begin{Shaded}
\begin{Highlighting}[]
\CommentTok{\# Header 1}
\CommentTok{\#\# Header 2}
\CommentTok{\#\#\# Header 3}
\end{Highlighting}
\end{Shaded}

\begin{itemize}
\tightlist
\item
  Use these section headings to indicate workflow.
\end{itemize}

\begin{Shaded}
\begin{Highlighting}[]
\CommentTok{\# Import packages and data}
\CommentTok{\# Tidy data}
\CommentTok{\# Wrangle data}
\CommentTok{\# Model data}
\CommentTok{\# Visualize data}
\end{Highlighting}
\end{Shaded}

\begin{itemize}
\item
  Press \texttt{ctrl\ +\ shift\ +\ o}. You can see a document outline based on these headings. This is a nice feature for finding code you need to focus.
\item
  If your project's scale is large, then divide these sections into files, number, and save them in \texttt{code} subdirectory.

  \begin{itemize}
  \tightlist
  \item
    01\_wrangling.Rmd
  \item
    02\_modeling.Rmd
    \ldots{}
  \end{itemize}
\end{itemize}

\hypertarget{making-a-project-computationally-reproducible}{%
\subsubsection{Making a project computationally reproducible}\label{making-a-project-computationally-reproducible}}

\begin{itemize}
\item
  \texttt{setwd()}: set a working directory.
\item
  Note that using \texttt{setwd()} is not a reproducible way to set up your project. For instance, none will be able to run the following code except me.
\end{itemize}

\begin{Shaded}
\begin{Highlighting}[]
\CommentTok{\# Set a working directory }
\KeywordTok{setwd}\NormalTok{(}\StringTok{"/home/jae/starwars"}\NormalTok{)}

\CommentTok{\# Do something }
\KeywordTok{ggplot}\NormalTok{(mtcars, }\KeywordTok{aes}\NormalTok{(}\DataTypeTok{x =}\NormalTok{ mpg, }\DataTypeTok{y =}\NormalTok{ wt)) }\OperatorTok{+}
\StringTok{   }\KeywordTok{geom\_point}\NormalTok{()}

\CommentTok{\# Export the object. }
\CommentTok{\# dot means the working directory set by setwd()}
\KeywordTok{ggsave}\NormalTok{(}\StringTok{"./outputs/example.png"}\NormalTok{) }\CommentTok{\# This is called relative path }
\end{Highlighting}
\end{Shaded}

\begin{itemize}
\item
  Instead, learn how to use \texttt{here()}'.

  \begin{itemize}
  \item
    Key idea: separate workflow (e.g., workspace information) from products (code and data). For more information, read Jenny Bryan's wonderful piece on \href{https://www.tidyverse.org/blog/2017/12/workflow-vs-script/}{project-oriented workflow}.
  \item
    Example
  \end{itemize}
\end{itemize}

\begin{Shaded}
\begin{Highlighting}[]
\CommentTok{\# New: Reproducible }

\KeywordTok{ggplot}\NormalTok{(mtcars, }\KeywordTok{aes}\NormalTok{(}\DataTypeTok{x =}\NormalTok{ mpg, }\DataTypeTok{y =}\NormalTok{ wt)) }\OperatorTok{+}
\StringTok{   }\KeywordTok{geom\_point}\NormalTok{()}

\KeywordTok{ggsave}\NormalTok{(}\KeywordTok{here}\NormalTok{(}\StringTok{"project"}\NormalTok{, }\StringTok{"outputs"}\NormalTok{, }\StringTok{"example.png"}\NormalTok{))}
\end{Highlighting}
\end{Shaded}

\begin{itemize}
\tightlist
\item
  How \texttt{here} works
\end{itemize}

\texttt{here()} function shows what's the top-level project directory.

\begin{Shaded}
\begin{Highlighting}[]
\NormalTok{here}\OperatorTok{::}\KeywordTok{here}\NormalTok{()}
\end{Highlighting}
\end{Shaded}

\begin{itemize}
\tightlist
\item
  Build a path including subdirectories
\end{itemize}

\begin{Shaded}
\begin{Highlighting}[]
\NormalTok{here}\OperatorTok{::}\KeywordTok{here}\NormalTok{(}\StringTok{"project"}\NormalTok{, }\StringTok{"outputs"}\NormalTok{)}
           \CommentTok{\#depth 1   \#depth 2}
\end{Highlighting}
\end{Shaded}

\begin{itemize}
\item
  How \texttt{here} defines the top-level project directory. The following list came from \href{https://github.com/jennybc/here_here}{the here package vignette}).

  \begin{itemize}
  \item
    Is a file named .here present?
  \item
    Is this an RStudio Project? (\textbf{Note that we already set up an RStudio Project!} So, if you use RStudio's project feature, then you are ready to use \texttt{here}.)
  \item
    Is this an R package? Does it have a DESCRIPTION file?
  \item
    Is this a remake project? Does it have a file named \texttt{remake.yml}?
  \item
    Is this a projectile project? Does it have a file named \texttt{.projectile}?
  \item
    Is this a checkout from a version control system? Does it have a directory named \texttt{.git} or \texttt{.svn}? Currently, only Git and Subversion are supported.
  \item
    If there's no match then use \texttt{set\_here()} to create an empty \texttt{.here} file.
  \end{itemize}
\end{itemize}

\textbf{Challenge 1}

\begin{enumerate}
\def\labelenumi{\arabic{enumi}.}
\tightlist
\item
  Can you define computational reproducibility?
\item
  Can you explain why sharing code and data is not enough for computational reproducibility?
\end{enumerate}

\hypertarget{version-control-git-and-bash}{%
\subsection{Version control (Git and Bash)}\label{version-control-git-and-bash}}

\includegraphics{https://github.com/dlab-berkeley/BashGit/raw/master/octobash.png}

\hypertarget{what-is-bash}{%
\subsubsection{What Is Bash?}\label{what-is-bash}}

\hypertarget{writing-your-first-shell-script}{%
\paragraph{Writing your first shell script}\label{writing-your-first-shell-script}}

Write a shell script that creates a directory called \texttt{/pdfs} under \texttt{/Download} directory, then find PDF files in \texttt{/Download} and copy those files to \texttt{pdfs}. This shell script creates a backup.

\begin{Shaded}
\begin{Highlighting}[]

\CommentTok{\#!/bin/sh}

\FunctionTok{mkdir}\NormalTok{ /home/jae/Downloads/pdfs }

\BuiltInTok{cd}\NormalTok{ Download}

\FunctionTok{cp}\NormalTok{ *.pdf pdfs/ }

\BuiltInTok{echo} \StringTok{"Copied pdfs"}
\end{Highlighting}
\end{Shaded}

\hypertarget{what-are-git-and-github}{%
\subsubsection{What Are Git and GitHub?}\label{what-are-git-and-github}}

\begin{figure}
\centering
\includegraphics{https://plain-text.co/figures/git-basic.png}
\caption{Figure 2.1. A schematic git workflow from Healy's ``The Plain Person's Guide to Plain Text Social Science''}
\end{figure}

\hypertarget{basics-git-push-and-git-pull}{%
\paragraph{\texorpdfstring{Basics: \texttt{git\ push} and \texttt{git\ pull}}{Basics: git push and git pull}}\label{basics-git-push-and-git-pull}}

\hypertarget{time-machine-git-revert}{%
\paragraph{\texorpdfstring{Time machine: \texttt{git\ revert}}{Time machine: git revert}}\label{time-machine-git-revert}}

\hypertarget{parallel-universe-git-branch}{%
\paragraph{\texorpdfstring{Parallel universe: \texttt{git\ branch}}{Parallel universe: git branch}}\label{parallel-universe-git-branch}}

\hypertarget{user-manual-readme}{%
\paragraph{\texorpdfstring{User-manual: \texttt{readme}}{User-manual: readme}}\label{user-manual-readme}}

\begin{itemize}
\tightlist
\item
  README.md
\end{itemize}

\includegraphics{https://libapps.s3.amazonaws.com/accounts/125446/images/README_Sample.png}

\begin{itemize}
\item
  In this simple markdown file, note some basic information about the project including the project structure.
\item
  This is how I used the \texttt{README.md} file for this course. Check out \href{https://github.com/jaeyk}{my GitHub account} to see how I manage my projects.
\end{itemize}

\hypertarget{deployment-github-pages}{%
\subsubsection{Deployment: GitHub Pages}\label{deployment-github-pages}}

\hypertarget{tracking-progress-github-issues}{%
\subsubsection{Tracking progress: GitHub Issues}\label{tracking-progress-github-issues}}

\hypertarget{project-management-github-dashboards}{%
\subsubsection{Project management: GitHub Dashboards}\label{project-management-github-dashboards}}

\hypertarget{writing-code-how-to-code-like-a-professional}{%
\section{Writing code: How to code like a professional}\label{writing-code-how-to-code-like-a-professional}}

\hypertarget{write-readable-code}{%
\subsection{Write readable code}\label{write-readable-code}}

\begin{itemize}
\tightlist
\item
  What is code style?
\end{itemize}

\begin{quote}
Every major open-source project has its own style guide: a set of conventions (sometimes arbitrary) about how to write code for that project. It is much easier to understand a large codebase when all the code in it is in a consistent style. - \href{https://google.github.io/styleguide/}{Google Style Guides}
\end{quote}

\begin{itemize}
\tightlist
\item
  Files
\item
  When naming files:

  \begin{itemize}
  \tightlist
  \item
    Don't use special characters. (Spaces make filenames awkward in the console/command-line.)
  \item
    Don't capitalize. (UNIX is case sensitive.)
  \item
    Numbering them if files should be run in an order.
  \end{itemize}
\end{itemize}

\begin{Shaded}
\begin{Highlighting}[]
\CommentTok{\# Good}
\NormalTok{fit\_models.R}

\CommentTok{\# Bad}
\NormalTok{fit models.R}
\end{Highlighting}
\end{Shaded}

\begin{itemize}
\tightlist
\item
  Objects
\item
  When naming objects:

  \begin{itemize}
  \tightlist
  \item
    Don't use special characters.
  \item
    Don't capitalize.
  \end{itemize}
\end{itemize}

\begin{Shaded}
\begin{Highlighting}[]
\CommentTok{\# Good }
\NormalTok{day\_one}
    
\CommentTok{\# Bad }
\NormalTok{DayOne}
\end{Highlighting}
\end{Shaded}

\begin{itemize}
\tightlist
\item
  Functions
\item
  When naming functions:

  \begin{itemize}
  \tightlist
  \item
    Don't use special characters.
  \item
    Don't capitalize.
  \item
    Use \texttt{verbs} instead of \texttt{nouns}. (Functions do something!)
  \end{itemize}
\end{itemize}

\begin{Shaded}
\begin{Highlighting}[]
\CommentTok{\# Good }
\NormalTok{run\_rdd }

\CommentTok{\# Bad }
\NormalTok{rdd}
\end{Highlighting}
\end{Shaded}

\begin{itemize}
\tightlist
\item
  Spacing
\end{itemize}

\begin{Shaded}
\begin{Highlighting}[]
\CommentTok{\# Good}
\NormalTok{x[, }\DecValTok{1}\NormalTok{] }

\KeywordTok{mean}\NormalTok{(x, }\DataTypeTok{na.rm =} \OtherTok{TRUE}\NormalTok{) }

\CommentTok{\# Bad}

\NormalTok{x[,}\DecValTok{1}\NormalTok{]}

\KeywordTok{mean}\NormalTok{ (x, }\DataTypeTok{na.rm =} \OtherTok{TRUE}\NormalTok{)}
\end{Highlighting}
\end{Shaded}

\begin{itemize}
\tightlist
\item
  Indenting
\end{itemize}

\begin{Shaded}
\begin{Highlighting}[]
\CommentTok{\# Good}
\ControlFlowTok{if}\NormalTok{ (y }\OperatorTok{<}\StringTok{ }\DecValTok{0}\NormalTok{) \{}
  \KeywordTok{message}\NormalTok{(}\StringTok{"y is negative"}\NormalTok{)}
\NormalTok{\}}

\CommentTok{\# Bad}
\ControlFlowTok{if}\NormalTok{ (y }\OperatorTok{<}\StringTok{ }\DecValTok{0}\NormalTok{) \{}
\KeywordTok{message}\NormalTok{(}\StringTok{"Y is negative"}\NormalTok{)\}}
\end{Highlighting}
\end{Shaded}

\begin{itemize}
\tightlist
\item
  Long lines
\end{itemize}

\begin{Shaded}
\begin{Highlighting}[]
\CommentTok{\# Good}
\KeywordTok{do\_something\_very\_complicated}\NormalTok{(}
  \DataTypeTok{something =} \StringTok{"that"}\NormalTok{,}
  \DataTypeTok{requires =}\NormalTok{ many,}
  \DataTypeTok{arguments =} \StringTok{"some of which may be long"}
\NormalTok{)}

\CommentTok{\# Bad}
\KeywordTok{do\_something\_very\_complicated}\NormalTok{(}\StringTok{"that"}\NormalTok{, }\DataTypeTok{requires =}\NormalTok{ many, }\DataTypeTok{arguments =}
                              \StringTok{"some of which may be long"}
\NormalTok{                              )}
\end{Highlighting}
\end{Shaded}

\begin{itemize}
\tightlist
\item
  Comments

  \begin{itemize}
  \tightlist
  \item
    Use comments to explain your decisions.
  \item
    But, show your code; Do not try to explain your code by comments.
  \item
    Also, try to comment out rather than delete the code that you experiment with.
  \end{itemize}
\end{itemize}

\begin{Shaded}
\begin{Highlighting}[]
\CommentTok{\# Average sleep hours of Jae}
\NormalTok{jae }\OperatorTok{\%>\%}
\StringTok{  }\CommentTok{\# By week}
\StringTok{  }\KeywordTok{group\_by}\NormalTok{(week) }\OperatorTok{\%>\%}
\StringTok{  }\CommentTok{\# Mean sleep hours }
\StringTok{  }\KeywordTok{summarise}\NormalTok{(}\DataTypeTok{week\_sleep =} \KeywordTok{mean}\NormalTok{(sleep, }\DataTypeTok{na.rm =} \OtherTok{TRUE}\NormalTok{))}
\end{Highlighting}
\end{Shaded}

\begin{itemize}
\tightlist
\item
  Pipes (chaining commands)
\end{itemize}

\begin{Shaded}
\begin{Highlighting}[]
\CommentTok{\# Good}
\NormalTok{iris }\OperatorTok{\%>\%}
\StringTok{  }\KeywordTok{group\_by}\NormalTok{(Species) }\OperatorTok{\%>\%}
\StringTok{  }\KeywordTok{summarize\_if}\NormalTok{(is.numeric, mean) }\OperatorTok{\%>\%}
\StringTok{  }\KeywordTok{ungroup}\NormalTok{() }\OperatorTok{\%>\%}
\StringTok{  }\KeywordTok{gather}\NormalTok{(measure, value, }\OperatorTok{{-}}\NormalTok{Species) }\OperatorTok{\%>\%}
\StringTok{  }\KeywordTok{arrange}\NormalTok{(value)}

\CommentTok{\# Bad}
\NormalTok{iris }\OperatorTok{\%>\%}\StringTok{ }\KeywordTok{group\_by}\NormalTok{(Species) }\OperatorTok{\%>\%}\StringTok{ }\KeywordTok{summarize\_all}\NormalTok{(mean) }\OperatorTok{\%>\%}
\NormalTok{ungroup }\OperatorTok{\%>\%}\StringTok{ }\KeywordTok{gather}\NormalTok{(measure, value, }\OperatorTok{{-}}\NormalTok{Species) }\OperatorTok{\%>\%}
\KeywordTok{arrange}\NormalTok{(value)}
\end{Highlighting}
\end{Shaded}

\begin{itemize}
\item
  Additional tips
\item
  Use \texttt{lintr} to check whether your code complies with a recommended style guideline (e.g., \texttt{tidyverse}) and \texttt{styler} package to format your code according to the style guideline.
\end{itemize}

\begin{figure}
\centering
\includegraphics{https://camo.githubusercontent.com/6cb80270269165a8d3046d2da03cbf2b8f19ee2f/687474703a2f2f692e696d6775722e636f6d2f61635632374e562e676966}
\caption{how lintr works}
\end{figure}

\hypertarget{write-reusable-code}{%
\subsection{Write reusable code}\label{write-reusable-code}}

\begin{itemize}
\tightlist
\item
  Pasting
\end{itemize}

\begin{quote}
Copy-and-paste programming, sometimes referred to as just pasting, is the production of highly repetitive computer programming code, as produced by copy and paste operations. It is primarily a pejorative term; those who use the term are often implying a lack of programming competence. It may also be the result of technology limitations (e.g., an insufficiently expressive development environment) as subroutines or libraries would normally be used instead. However, there are occasions when copy-and-paste programming is considered acceptable or necessary, such as for boilerplate, loop unrolling (when not supported automatically by the compiler), or certain programming idioms, and it is supported by some source code editors in the form of snippets. - \href{https://en.wikipedia.org/wiki/Copy-and-paste_programming}{Wikipedia}
\end{quote}

\begin{itemize}
\item
  It's okay for pasting for the first attempt to solve a problem. But if you copy and paste three times (a.k.a. \href{https://en.wikipedia.org/wiki/Rule_of_three_(computer_programming)}{Rule of Three} in programming), something's wrong. You're working too hard. You need to be lazy. What do I mean and how can you do that?
\item
  Example
\item
  Let's imagine \texttt{df} is a survey dataset.

  \begin{itemize}
  \item
    \texttt{a,\ b,\ c,\ d} = Survey questions
  \item
    \texttt{-99}: non-responses
  \item
    Your goal: replace \texttt{-99} with \texttt{NA}
  \end{itemize}
\end{itemize}

\begin{Shaded}
\begin{Highlighting}[]
\CommentTok{\# Data}

\KeywordTok{set.seed}\NormalTok{(}\DecValTok{1234}\NormalTok{) }\CommentTok{\# for reproducibility }

\NormalTok{df <{-}}\StringTok{ }\KeywordTok{tibble}\NormalTok{(}\StringTok{"a"}\NormalTok{ =}\StringTok{ }\KeywordTok{sample}\NormalTok{(}\KeywordTok{c}\NormalTok{(}\OperatorTok{{-}}\DecValTok{99}\NormalTok{, }\DecValTok{1}\OperatorTok{:}\DecValTok{3}\NormalTok{), }\DataTypeTok{size =} \DecValTok{5}\NormalTok{ , }\DataTypeTok{replace=} \OtherTok{TRUE}\NormalTok{),}
             \StringTok{"b"}\NormalTok{ =}\StringTok{ }\KeywordTok{sample}\NormalTok{(}\KeywordTok{c}\NormalTok{(}\OperatorTok{{-}}\DecValTok{99}\NormalTok{, }\DecValTok{1}\OperatorTok{:}\DecValTok{3}\NormalTok{), }\DataTypeTok{size =} \DecValTok{5}\NormalTok{ , }\DataTypeTok{replace=} \OtherTok{TRUE}\NormalTok{),}
             \StringTok{"c"}\NormalTok{ =}\StringTok{ }\KeywordTok{sample}\NormalTok{(}\KeywordTok{c}\NormalTok{(}\OperatorTok{{-}}\DecValTok{99}\NormalTok{, }\DecValTok{1}\OperatorTok{:}\DecValTok{3}\NormalTok{), }\DataTypeTok{size =} \DecValTok{5}\NormalTok{ , }\DataTypeTok{replace=} \OtherTok{TRUE}\NormalTok{),}
             \StringTok{"d"}\NormalTok{ =}\StringTok{ }\KeywordTok{sample}\NormalTok{(}\KeywordTok{c}\NormalTok{(}\OperatorTok{{-}}\DecValTok{99}\NormalTok{, }\DecValTok{1}\OperatorTok{:}\DecValTok{3}\NormalTok{), }\DataTypeTok{size =} \DecValTok{5}\NormalTok{ , }\DataTypeTok{replace=} \OtherTok{TRUE}\NormalTok{))}
\end{Highlighting}
\end{Shaded}

\begin{Shaded}
\begin{Highlighting}[]
\CommentTok{\# Copy and paste }
\NormalTok{df}\OperatorTok{$}\NormalTok{a[df}\OperatorTok{$}\NormalTok{a }\OperatorTok{==}\StringTok{ }\DecValTok{{-}99}\NormalTok{] <{-}}\StringTok{ }\OtherTok{NA}
\NormalTok{df}\OperatorTok{$}\NormalTok{b[df}\OperatorTok{$}\NormalTok{b }\OperatorTok{==}\StringTok{ }\DecValTok{{-}99}\NormalTok{] <{-}}\StringTok{ }\OtherTok{NA}
\NormalTok{df}\OperatorTok{$}\NormalTok{c[df}\OperatorTok{$}\NormalTok{c }\OperatorTok{==}\StringTok{ }\DecValTok{{-}99}\NormalTok{] <{-}}\StringTok{ }\OtherTok{NA}
\NormalTok{df}\OperatorTok{$}\NormalTok{d[df}\OperatorTok{$}\NormalTok{d }\OperatorTok{==}\StringTok{ }\DecValTok{{-}99}\NormalTok{] <{-}}\StringTok{ }\OtherTok{NA}

\NormalTok{df}
\end{Highlighting}
\end{Shaded}

\begin{verbatim}
## # A tibble: 5 x 4
##       a     b     c     d
##   <dbl> <dbl> <dbl> <dbl>
## 1     3     3     3     1
## 2     3     2     3     1
## 3     1    NA     1     2
## 4     1    NA     2     1
## 5    NA     1     1     3
\end{verbatim}

\begin{itemize}
\tightlist
\item
  Using a function

  \begin{itemize}
  \tightlist
  \item
    function: input + computation + output
  \item
    If you write a function, you gain efficiency because you don't need to copy and paste the computation part.
  \end{itemize}
\end{itemize}

\begin{Shaded}
\begin{Highlighting}[]
\CommentTok{\# Create a custom function}
\NormalTok{fix\_missing <{-}}\StringTok{ }\ControlFlowTok{function}\NormalTok{(x) \{ }\CommentTok{\# INPUT}
\NormalTok{  x[x }\OperatorTok{==}\StringTok{ }\DecValTok{{-}99}\NormalTok{] <{-}}\StringTok{ }\OtherTok{NA} \CommentTok{\# COMPUTATION}
\NormalTok{  x }\CommentTok{\# OUTPUT }
\NormalTok{\}}

\CommentTok{\# Apply the function to each column (vector)}
\CommentTok{\# This iterated part can and should be automated.}
\NormalTok{df}\OperatorTok{$}\NormalTok{a <{-}}\StringTok{ }\KeywordTok{fix\_missing}\NormalTok{(df}\OperatorTok{$}\NormalTok{a)}
\NormalTok{df}\OperatorTok{$}\NormalTok{b <{-}}\StringTok{ }\KeywordTok{fix\_missing}\NormalTok{(df}\OperatorTok{$}\NormalTok{b)}
\NormalTok{df}\OperatorTok{$}\NormalTok{c <{-}}\StringTok{ }\KeywordTok{fix\_missing}\NormalTok{(df}\OperatorTok{$}\NormalTok{c)}
\NormalTok{df}\OperatorTok{$}\NormalTok{d <{-}}\StringTok{ }\KeywordTok{fix\_missing}\NormalTok{(df}\OperatorTok{$}\NormalTok{d)}

\NormalTok{df}
\end{Highlighting}
\end{Shaded}

\begin{itemize}
\tightlist
\item
  Automation

  \begin{itemize}
  \tightlist
  \item
    Many options for automation in R: \texttt{for\ loop}, \texttt{apply} family, etc.
  \item
    Here's a tidy solution comes from \texttt{purrr} package.
  \item
    The power and joy of one-liner.
  \end{itemize}
\end{itemize}

\begin{Shaded}
\begin{Highlighting}[]
\NormalTok{df <{-}}\StringTok{ }\NormalTok{purrr}\OperatorTok{::}\KeywordTok{map\_df}\NormalTok{(df, fix\_missing) }\CommentTok{\# What is this magic? We will unpack the blackbox (\textasciigrave{}map\_df()\textasciigrave{}) later.}

\NormalTok{df}
\end{Highlighting}
\end{Shaded}

\begin{itemize}
\item
  Takeaways

  \begin{enumerate}
  \def\labelenumi{\arabic{enumi}.}
  \item
    Your code becomes more reusable, when it's easier to \textbf{change, debug, and scale up}. Don't repeat yourself and embrace the power of lazy programming.
  \item
    Only when your code becomes \textbf{reusable}, you would become \textbf{efficient} in your data work. Otherwise, you need to start from scratch or copy and paste, when you work on a new project.
  \end{enumerate}
\end{itemize}

\begin{quote}
Lazy, because only lazy programmers will want to write the kind of tools that might replace them in the end. Lazy, because only a lazy programmer will avoid writing monotonous, repetitive code---thus avoiding redundancy, the enemy of software maintenance and flexible refactoring. Mostly, the tools and processes that come out of this endeavor fired by laziness will speed up the production. - \href{http://blogoscoped.com/archive/2005-08-24-n14.html}{Philipp Lenssen}
\end{quote}

\hypertarget{test-your-code-systematically}{%
\subsection{Test your code systematically}\label{test-your-code-systematically}}

\hypertarget{asking-questions-minimal-reproducible-example}{%
\section{Asking questions: Minimal reproducible example}\label{asking-questions-minimal-reproducible-example}}

\hypertarget{how-to-create-a-minimal-reproducible-example}{%
\subsection{How to create a minimal reproducible example}\label{how-to-create-a-minimal-reproducible-example}}

\textbf{References}

\begin{itemize}
\item
  Project-oriented research

  \begin{itemize}
  \item
    Computational reproducibility

    \begin{itemize}
    \item
      \href{https://github.com/swcarpentry/good-enough-practices-in-scientific-computing/blob/gh-pages/good-enough-practices-for-scientific-computing.pdf}{``Good Enough Practices in Scientific Computing''} by PLOS
    \item
      \href{https://swcarpentry.github.io/r-novice-gapminder/02-project-intro/}{Project Management with RStudio} by Software Carpentry
    \item
      \href{https://kbroman.org/steps2rr/}{Initial steps toward reproducible research} by Karl Broman
    \end{itemize}
  \item
    Version control

    \begin{itemize}
    \item
      \href{https://swcarpentry.github.io/git-novice/}{Version Control with Git} by Software Carpentry
    \item
      \href{http://plain-text.co/}{The Plain Person's Guide to Plain Text Social Science} by Kieran Healy
    \end{itemize}
  \end{itemize}
\item
  Writing code

  \begin{itemize}
  \tightlist
  \item
    Style guides

    \begin{itemize}
    \tightlist
    \item
      R

      \begin{itemize}
      \tightlist
      \item
        \href{https://google.github.io/styleguide/Rguide.xml}{Google's R style guide}
      \item
        \href{http://r-pkgs.had.co.nz/r.html}{R code style guide} by Hadley Wickham
      \item
        \href{http://style.tidyverse.org/}{The tidyverse style guide} by Hadley Wickham
      \end{itemize}
    \item
      Python

      \begin{itemize}
      \tightlist
      \item
        \href{https://github.com/google/styleguide/blob/gh-pages/pyguide.md}{Google Python Style Guide}
      \item
        \href{https://docs.python-guide.org/writing/style/\#zen-of-python}{Code Style} by the Hitchhiker's Guide to Python
      \end{itemize}
    \end{itemize}
  \end{itemize}
\item
  Asking questions
\end{itemize}

\hypertarget{tidy_data}{%
\chapter{Tidy data and its friends}\label{tidy_data}}

\hypertarget{tidy-data-and-why-it-matters}{%
\section{Tidy data and why it matters}\label{tidy-data-and-why-it-matters}}

\hypertarget{wrangling-data}{%
\section{Wrangling data}\label{wrangling-data}}

\hypertarget{how-to-wrangle-data}{%
\section{How to wrangle data}\label{how-to-wrangle-data}}

\hypertarget{how-to-wrangle-data-at-scale}{%
\section{How to wrangle data at scale}\label{how-to-wrangle-data-at-scale}}

\hypertarget{modeling-and-visualizing-tidy-data}{%
\section{Modeling and visualizing tidy data}\label{modeling-and-visualizing-tidy-data}}

\hypertarget{functional_programming}{%
\chapter{Automating repeated things}\label{functional_programming}}

\hypertarget{from-for-loop-to-functional-programing}{%
\section{From for loop to functional programing}\label{from-for-loop-to-functional-programing}}

\hypertarget{developing-your-own-data-tools}{%
\section{Developing your own data tools}\label{developing-your-own-data-tools}}

\hypertarget{semi_structured_data}{%
\chapter{Semi-structured data}\label{semi_structured_data}}

\hypertarget{htmlcss-web-scraping}{%
\section{HTML/CSS: web scraping}\label{htmlcss-web-scraping}}

\hypertarget{xmljson-social-media-scraping}{%
\section{XML/JSON: social media scraping}\label{xmljson-social-media-scraping}}

\hypertarget{machine_learning}{%
\chapter{High-dimensional data}\label{machine_learning}}

\hypertarget{supervised-machine-learning}{%
\section{Supervised machine learning}\label{supervised-machine-learning}}

\hypertarget{regularization}{%
\subsection{Regularization}\label{regularization}}

\hypertarget{decision-tree-and-ensemble-models}{%
\subsection{Decision tree and ensemble models}\label{decision-tree-and-ensemble-models}}

\hypertarget{unsupervised-machine-learning}{%
\section{Unsupervised machine learning}\label{unsupervised-machine-learning}}

\hypertarget{dimension-reduction}{%
\subsection{Dimension reduction}\label{dimension-reduction}}

\hypertarget{clustering}{%
\subsection{Clustering}\label{clustering}}

\hypertarget{big_data}{%
\chapter{Big data}\label{big_data}}

\hypertarget{database-and-sql}{%
\section{Database and SQL}\label{database-and-sql}}

  \bibliography{book.bib,packages.bib}

\end{document}
